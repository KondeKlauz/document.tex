\documentclass{article}
\usepackage[T1]{fontenc}
\usepackage[brazilian]{babel}
\usepackage[top = 2cm, bottom = 2cm, left = 2.5cm, right = 2.5cm]{geometry}
\begin{document}
	\title{\textbf{Curso de LaTeX}}
	\author{RAFAEL GOMES DA SILVA}
	\date{\underline{19 de Fevereiro de 1997}}
	\maketitle
	Última obra escrita por Lispector, Um sopro de vida (Pulsações) começou a ser elaborada em 1974, época em que a escritora estava gravemente doente, terminando por falecer em dezembro de 1977. Publicado postumamente, o livro resulta de três anos de escrita, desenvolvida concomitantemente à da novela A hora da estrela, ambas narrativas fortemente tomadas pela música. “Não consigo imaginar uma vida sem a arte de escrever ou de pintar ou de fazer música.” Em Um sopro de vida, a presença musical tem a ver com a busca da expressão capaz de chegar a zonas dificilmente traduzíveis por palavras.
	
	O enredo é simples: o narrador-escritor, ao escrever sobre Angela Pralini, personagem de outras narrativas da escritora, se vê diante de um espelho invertido de si próprio. Angela, também escritora, o expõe aos próprios desejos inalcançados e traz a discussão do tênue limite entre autor e personagem. A situação provoca reflexões sobre o estar no mundo e sobre o processo criativo. Na “luta entre o ser e o existir”, instala-se uma atmosfera sensível, invadida aqui e ali pela melancolia, sem deixar de enfrentar questões essenciais para os que querem da vida muito mais do que o previsível. O narrador anuncia nas primeiras páginas: “Escrevo como se fosse para salvar a vida de alguém. Provavelmente a minha própria vida. Viver é uma espécie de loucura que a morte faz. Vivam os mortos porque nele vivemos.”.
	
	Além de doação, a escrita se apresenta como um vaticínio: “Minha vida me quer escritor e então escrevo. Não é por escolha: é íntima voz de comando”.
	
	Destino que implica um risco: “Escrever pode tornar a pessoa louca… Tenho medo de minha liberdade…”. No impasse de a palavra poder ou não significar o que se pensa e o que se sente, a obra apresenta a dualidade do autor-narrador, dividido entre racionalidade, instinto, corpo e liberdade. Angela, retratada como escritora em crise criativa, experimenta a profundidade do silêncio: “O dia corre lá fora à toa e há abismos de silêncio em mim”, A paixão e a dor da escrita são mostradas em carne viva. Lispector retoma um dos topoi de sua obra, os processos e modelos narrativos: seguir a via dos que obedecem à história estruturada e lógica; ou escolher o perturbador caminho do inconsciente. “Devo-me interessar pelo acontecimento? Será que desço tanto a ponto de encher as páginas com informações sobre os atos? Devo imaginar uma história ou dou largas à inspiração caótica?” Um leitor sensível haverá de se permitir acompanhar estas pulsações, indicadas no subtítulo da obra.
	
\end{document}